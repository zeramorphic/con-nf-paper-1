\section{Quivers and paths}

\begin{definition}
    A \mdef{Combinatorics/Quiver/Basic}{Quiver}{quiver} on a type \( \alpha \) of vertices assigns to every pair \( x, y : \alpha \) of vertices a type \( \symrm{Hom}(x, y) \) of arrows from \( x \) to \( y \).
\end{definition}
\begin{definition}
    A \mdef{Combinatorics/Quiver/Path}{Quiver.Path}{path} in a quiver between two vertices \( x, y : \alpha \) is a finite list of vertices beginning with \( x \) and ending with \( y \), connecting each pair of adjacent vertices \( a, b \) with an element of \( \symrm{Hom}(a, b) \).
    The type of such paths is written \( x \rightsquigarrow y \).
    The empty path is written \( \varnothing : x \rightsquigarrow x \).
    The \mdef{Combinatorics/Quiver/Path}{Quiver.Path.comp}{composition} of paths \( p : x \rightsquigarrow y, q : y \rightsquigarrow z \) is denoted by \( p \gg q : x \rightsquigarrow z \).
\end{definition}
\begin{remark}
    In mathlib, paths are defined as an inductive type.
    If there is exactly one morphism in a given hom-set \( \symrm{Hom}(a, b) \), it is denoted \( a \to b \).
    We will implicitly convert morphisms \( e : \symrm{Hom}(a, b) \) to their \mdef{Combinatorics/Quiver/Path}{Quiver.Hom.toPath}{corresponding paths} \( e : a \rightsquigarrow b \).
\end{remark}
\begin{definition}
    The \mdef{Combinatorics/Quiver/Path}{Quiver.Path.length}{length} of a path is the number of arrows in that path, or exactly one less than the number of vertices in the list.
\end{definition}
