\subsection{Near-litter actions}

\begin{definition}
    A \cdef{Foa/Action/NearLitterAction}{NearLitterAction}{near-litter action} is a pair \( \psi = \langle \psi^A, \psi^L \rangle \) where \( \psi^A : \mathcal A \rightdasharrow \mathcal A \) and \( \psi^L : \mathcal L \rightdasharrow \mathcal N \), such that \( \dom \psi^A \) and \( \dom \psi^L \) are small.
\end{definition}
We let near-litter actions act on all atoms and litters, where the action on a litter yields a near-litter.
If the atom or litter does not lie in the domain, its value is irrelevant.
\begin{definition}
    \label{def:NearLitterAction.Lawful}
    A near-litter action is \cdef{Foa/Action/NearLitterAction}{NearLitterAction.Lawful}{lawful} if
    \begin{enumerate}
        \item \( \psi^A \) is injective;
        \item \( \psi^L \) is injective in the sense that if \( \psi\ L_1 \cap \psi\ L_2 \neq \varnothing \) then \( L_1 = L_2 \);
        \item for \( a \in \dom \psi^A \) and \( L \in \dom \psi^L \), \( a^\circ = L \Leftrightarrow \psi\ a \in \psi\ L \).
    \end{enumerate}
\end{definition}
\begin{definition}
    Let \( \psi \) be a near-litter action.
    A litter \( L \) is \cdef{Foa/Action/NearLitterAction}{NearLitterAction.BannedLitter}{\( \psi \)-banned} if one of the following holds.
    \begin{enumerate}
        \item \( L = a^\circ \) for some \( a \in \dom \psi^A \);
        \item \( L \in \dom \psi^L \);
        \item \( L = (\psi\ a)^\circ \) for some \( a \in \dom \psi^A \);
        \item \( L = (\psi\ L')^\circ \) for some \( L' \in \dom \psi^L \);
        \item \( L = a^\circ \) for some \( a : \mathcal A \) with a litter \( L' \in \dom \psi^L \) such that \( a \in \psi\ L' \setminus \mathcal A_{(\psi\ L')^\circ} \).
    \end{enumerate}
\end{definition}
\begin{lemma}
    Let \( L \in \dom \psi^L \).
    Then for all \( a \in \psi\ L \), \( a^\circ \) is \( \psi \)-banned.
\end{lemma}
\begin{proof}
    Either case (iv) or (v) will apply for each atom \( a \in \psi\ L \).
\end{proof}
\begin{lemma}
    \label{lem:bannedLitter_small}
    The set of \( \psi \)-banned litters is small.
    In particular, the set of non-\( \psi \)-banned litters has cardinality \( \#\mu \).
\end{lemma}
\begin{proof}
    We show that each constructor gives rise to only a small amount of litters.
    Cases (i) and (iii) follow as \( \dom \psi^A \) is small; cases (ii) and (iv) follow as \( \dom \psi^L \) is small.
    For case (v), first note that the following set is small by \cref{lem:small}(vii), giving the result.
    \[ \bigcup_{L \in \dom \psi^L} \psi\ L \setminus \mathcal A_{(\psi\ L)^\circ} \]
\end{proof}
\begin{definition}
    We define a partial order structure on partial functions of type \( \alpha \rightdasharrow \beta \): we say that \( f \leq g \) if \( \dom f \subseteq \dom g \) and \( f\ x = g\ x \) for all \( x \in \dom f \).
\end{definition}
\begin{definition}
    We define a partial order on near-litter actions: \( \psi_1 \leq \psi_2 \) if \( \psi_1^A \leq \psi_2^A \) and \( \psi_1^L \leq \psi_2^L \).
\end{definition}
\begin{lemma}
    \label{lem:NearLitterAction.Lawful.le}
    If \( \psi_1 \leq \psi_2 \) and \( \psi_2 \) is lawful, then \( \psi_1 \) is lawful.
\end{lemma}
\begin{proof}
    By combining the definitions.
\end{proof}
\begin{definition}
    Let \( \psi \) be a near-litter action and \( L \in \dom \psi^L \).
    We say that \( \psi \) is \cdef{Foa/Action/NearLitterAction}{NearLitterAction.PreciseAt}{precise at \( L \)} if
    \begin{enumerate}
        \item \( \psi\ L \symmdiff \mathcal A_{(\psi\ L)^\circ} \subseteq \ran \psi^A \);
        \item \( \ran \psi^A \cap \mathcal A_L \subseteq \dom \psi^A \);
        \item \( \dom \psi^A \cap \psi\ L \subseteq \ran \psi^A \).
    \end{enumerate}
    We say that \( \psi \) is \cdef{Foa/Action/NearLitterAction}{NearLitterAction.Precise}{precise} if it is precise at every litter in its domain.
\end{definition}
\begin{definition}
    Let \( \beta < \alpha \) and \( A : \beta \rightsquigarrow \bot \), and suppose \( \psi \) is a lawful near-litter action.
    The \cdef{Foa/Action/NearLitterAction}{NearLitterAction.flexibleLitterLocalPerm}{\( A \)-flexible litter local permutation} of \( \psi \) is a local permutation of litters obtained by \cref{lem:LocalPerm.complete} that agrees with \( \psi^L \) on all \( A \)-flexible litters in \( \dom \psi^L \), where the sandbox is some set of non-banned \( A \)-flexible litters.
\end{definition}
\begin{lemma}
    The preconditions for \cref{lem:LocalPerm.complete} required in the above definition are satisfied.
\end{lemma}
\begin{proof}
    We need to check assumptions (iv)--(vi).
    \begin{enumerate}
        \setcounter{enumi}{3}
        \item We show that there are \( \#\mu \) non-banned \( A \)-flexible litters.
        Suppose there were less than \( \#\mu \) such litters.
        As the set of banned litters is small by \cref{lem:bannedLitter_small}, there would then be less than \( \#\mu \) \( A \)-flexible litters, contradicting \cref{lem:mk_flexible}.
        \item Every litter in \( \dom \psi^L \cup \psi '' \dom \psi^L \) is banned, and the sandbox is specified to contain only non-banned litters.
        \item Holds as \( \psi \) is lawful.
    \end{enumerate}
\end{proof}
\begin{lemma}
    \label{lem:dom_flexibleLitterLocalPerm_small}
    The domain of the \( A \)-flexible litter local permutation is small.
\end{lemma}
\begin{proof}
    Follows from cardinal arithmetic after unfolding the definition of the sandbox subset as \( \dom \psi^L \) is small.
\end{proof}
