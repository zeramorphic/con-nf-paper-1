\subsection{Flexibility of litters}

\begin{definition}
    Let \( A \) be a \( \beta \)-extended type index and \( L \) a litter.
    We say that \( L \) is \cdef{Foa/Basic/Flexible}{Inflexible}{\( A \)-inflexible} if either
    \begin{enumerate}
        \item there exist \( \gamma, \delta, \varepsilon : \lambda \) with \( \delta, \varepsilon < \gamma \) and \( \delta \neq \varepsilon \), and a path \( B : \beta \rightsquigarrow \gamma \) and tangle \( t : \tau_\delta \), such that
        \[ A = B \gg (\gamma \to \varepsilon) \gg (\varepsilon \to \bot);\quad L = f_{\delta,\varepsilon}\ t \]
        or
        \item there exist \( \gamma, \varepsilon : \lambda \) with \( \varepsilon < \gamma \), and a path \( B : \beta \rightsquigarrow \gamma \) and atom \( a \), such that
        \[ A = B \gg (\gamma \to \varepsilon) \gg (\varepsilon \to \bot);\quad L = f_{\bot,\varepsilon}\ a \]
    \end{enumerate}
    We call (i) the \emph{proper} case and (ii) the \emph{base} case.
    A litter which is not \( A \)-inflexible is called \cdef{Foa/Basic/Flexible}{Flexible}{\( A \)-flexible}.
\end{definition}
\begin{lemma}
    \label{lem:mk_flexible}
    Let \( A \) be a \( \beta \)-extended type index.
    Then there are exactly \( \#\mu \) \( A \)-flexible litters.
\end{lemma}
\begin{proof}
    Since there are \( \#\mu \) litters by \cref{lem:mk_litter}, we need only show that there are at least \( \#\mu \) such litters.
    To each \( \nu : \mu \) we assign the litter \( \langle \nu, \bot, \alpha \rangle \).
    Clearly this assignment is injective.
    Since \( \varepsilon < \alpha \) in each case of inflexibility, these litters must be \( A \)-flexible.
\end{proof}
\begin{lemma}
    \label{lem:comp_flexible}
    If \( L \) is \( A \)-inflexible, then \( L \) is \( B \gg A \)-inflexible.
    Conversely, if \( L \) is \( B \gg A \)-flexible, \( L \) is \( A \)-flexible.
\end{lemma}
\begin{proof}
    Case checking.
\end{proof}
\begin{definition}
    \cdef{Foa/Basic/Flexible}{InflexibleCoePath}{Proper \( A \)-inflexible path data} is a tuple \( \langle \gamma, \delta, \varepsilon, B \rangle \) where
    \[ \delta, \varepsilon < \gamma;\quad \delta \neq \varepsilon; \quad A = B \gg (\gamma \to \varepsilon) \gg (\varepsilon \to \bot) \]
    \cdef{Foa/Basic/Flexible}{InflexibleBotPath}{Base \( A \)-inflexible path data} is a tuple \( \langle \gamma, \varepsilon, B \rangle \) where
    \[ \varepsilon < \gamma; \quad A = B \gg (\gamma \to \varepsilon) \gg (\varepsilon \to \bot) \]
\end{definition}
\begin{definition}
    \cdef{Foa/Basic/Flexible}{InflexibleCoe}{Proper \( \langle A, L \rangle \)-inflexible data} is a pair \( \langle D, t \rangle \) where \( D \) is proper \( A \)-inflexible path data and \( L = f_{\delta,\varepsilon}\ t \).
    \cdef{Foa/Basic/Flexible}{InflexibleBot}{Base \( \langle A, L \rangle \)-inflexible data} is a pair \( \langle D, a \rangle \) where \( D \) is base \( A \)-inflexible path data and \( L = f_{\bot,\varepsilon}\ a \).
\end{definition}
\begin{lemma}
    A litter can be \( A \)-inflexible in precisely one way; that is, the types of proper and base \( \langle A, L \rangle \)-inflexible data are jointly a subsingleton.
\end{lemma}
\begin{proof}
    Use \cref{lem:fuzz_congr,lem:fuzz_injective}.
\end{proof}
