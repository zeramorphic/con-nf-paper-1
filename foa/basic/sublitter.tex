\subsection{Sublitters}
\begin{definition}
    A \emph{sublitter} is a litter \( L \) and a set of atoms \( s \) such that \( s \subseteq \mathcal A_L \), and \( \mathcal A_L \setminus s \) is small.
\end{definition}
A sublitter has a natural coercion into a set of atoms, and into a near-litter.
\begin{lemma}
    \label{lem:mk_sublitter}
    The cardinality of any sublitter is \( \#\kappa \).
\end{lemma}
\begin{proof}
    Follows directly from \cref{lem:mk_nearLitter''}.
\end{proof}
\begin{lemma}
    \label{lem:Sublitter.equiv}
    Let \( S, T \) be sublitters.
    Then there is an equivalence \( S \simeq T \).
    For each such pair we make a choice of equivalence denoted \( \pi_{S, T} : S \simeq T \).
\end{lemma}
\begin{proof}
    Both have cardinality \( \#\kappa \) by \cref{lem:mk_sublitter}.
\end{proof}
