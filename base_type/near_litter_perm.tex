\section{Near-litter permutations}

\begin{definition}
    A \cdef{BaseType/NearLitterPerm}{NearLitterPerm}{near-litter permutation} is a pair \( \pi = \langle \pi^A, \pi^L \rangle \) where \( \pi^A : \symsf{Perm}\ \mathcal A \) and \( \pi^L : \symsf{Perm}\ \mathcal L \), such that if \( s \) is a near-litter to \( L \), \( {\pi^A} '' s \) is a near-litter to \( \pi^L\ L \).
    Thus a near-litter permutation induces a permutation of near-litters.
    The type of near-litter permutations is denoted \( \mathcal P \).
\end{definition}
We suppress the superscripts on near-litter permutations and use function application syntax for the action of a near-litter permutation on atoms, litters, and near-litters: for example, \( \pi^A\ a = \pi\ a \).
Note that the action on litters is `rough': we map litters to litters and not near-litters.
If the precise image of a litter \( L \) under a permutation \( \pi \) is desired, it can be obtained using \( \pi\ (\symsf{NL}\ L) \).
\begin{lemma}
    \label{lem:NearLitterPerm.atomPerm_injective}
    If the atom permutations of two near-litter permutations agree, then the permutations are equal.
\end{lemma}
\begin{proof}
    Let \( L : \mathcal L \) and \( \pi, \pi' \) be near-litter permutations.
    The values of \( \pi\ (\symsf{NL}\ L) \) and \( \pi'\ (\symsf{NL}\ L) \) depend only on the atom maps in question.
    The result then follows from \cref{lem:isNearLitter}(iv).
\end{proof}
\begin{lemma}
    The near-litter permutations form a group with identity \( \symrm{id} \) and operation \( \circ \).
\end{lemma}
\begin{lemma}
    \label{lem:smul_nearLitter_eq_smul_symmDiff_smul}
    Let \( \pi \) be a near-litter permutation and let \( N \) be a near-litter.
    Then, the following equality of sets holds.
    \[ \pi\ N = (\pi\ (\mathsf{NL}\ N^\circ)) \symmdiff (\pi '' (\mathcal A_{N^\circ} \symmdiff N)) \]
\end{lemma}
\begin{proof}
    After applying set extensionality, this proof becomes simple case checking.
\end{proof}
